\makeatletter

%----------------------------
%	Zauberzeug
\newcounter{ZauberAnzahl}
\def\ZauberTabelleInhaltA{}
\def\ZauberTabelleInhaltB{}

% neuen Zauber hinzufügen
% Argumente sind:
%	 1: Zaubername
%	 2: Probe Eig. 1
%	 3: Probe Eig. 2
%	 4: Probe Eig. 3
%	 5: ZfW
%	 6: ZD
%	 7: Kosten
%	 8: Reichweite
%	 9: WD
%	10: Anmerkungen
%	11: Komplexität
%	12: Repräsentation
%	13: Hauszauber?
%	14: Lern
\newcommand{\neuerZauber}[9]{%
\addtocounter{ZauberAnzahl}{1}
\ifnum\value{ZauberAnzahl}>32\g@addto@macro\ZauberTabelleInhaltB{ & #1 & #2 $\bullet$ #3 $\bullet$ #4 & #5 & & #6 & #7 & #8 & #9 &}
\else\g@addto@macro\ZauberTabelleInhaltA{ & #1 & #2 $\bullet$ #3 $\bullet$ #4 & #5 & & #6 & #7 & #8 & #9 &}
\fi
\neuerZauberCont
}
\newcommand{\neuerZauberCont}[5]{%
\ifnum\value{ZauberAnzahl}>32\g@addto@macro\ZauberTabelleInhaltB{ #1 & #2 & #3 & #4 & #5\\\hline}
\else\g@addto@macro\ZauberTabelleInhaltA{ #1 & #2 & #3 & #4 & #5\\\hline}
\fi
}

\makeatother
